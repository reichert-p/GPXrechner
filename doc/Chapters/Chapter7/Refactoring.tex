\section{Kapitel 7: Refactoring}

\subsection{Code Smells}

\subsubsection{Code Smell: Duplicated Code}

%(Ausgebessert in commit https://github.com/reichert-p/GPXrechner/commit/4039bc31f6310508875c59ed44b6b227f2f0510d )

Um Lösungen zu bewerten, muss bei Bitstringcodierung (welche in der Klasse Representation verwendet wird) gefiltert werden welche Umwege tatsächlich genommen werden.

Vorheriger Zustand:

\begin{lstlisting}
// in Klasse StayNightEvaluation 

private double getWeightedOvershoot(ArrayList<Location> path, Detours detours, Representation representation) {
  List<Detours.Detour> visitedDetours = new ArrayList<>();
  for (int i = 0; i < detours.getPossibleDetours().size();i++){
    if (representation.getBitstring()[i]){
      visitedDetours.add(detours.getPossibleDetours().get(i));
    }
  }
    List<Detours.Detour> orderedVisitedDetours = visitedDetours.stream().sorted(Comparator.comparing(Detours.Detour::getPosition)).toList();
...
}

\end{lstlisting}

\begin{lstlisting}
// in Klasse SupplyEvaluation 
	
private double getOvershoot(ArrayList<Location> path, Detours detours, Representation representation) {
  List<Detours.Detour> visitedDetours = new ArrayList<>();
  for (int i = 0; i < detours.getPossibleDetours().size();i++){
    if (representation.getBitstring()[i]){
      visitedDetours.add(detours.getPossibleDetours().get(i));
    }
  }
List<Detours.Detour> orderedVisitedDetours = visitedDetours.stream().sorted(Comparator.comparing(Detours.Detour::getPosition)).toList();
...	
}
	
\end{lstlisting}

Der in Commit \href{commit https://github.com/reichert-p/GPXrechner/commit/4039bc31f6310508875c59ed44b6b227f2f0510d}{4039bc3} gewählte Lösungsweg lagert die duplizierte Funktionalität aus in die externe statische Klasse EvaluationHelper. In den einzelnen Bewertungsfunktionen wird die Funktionalität aus der EvaluationHelper Klasse aufgerufen.

\begin{lstlisting}

// in der Klasse EvaluationHelper 

public static List<Detours.Detour> getRepresentedDetoursOrdered(Detours detours, Representation representation){
  List<Detours.Detour> visitedDetours = new ArrayList<>();
  for (int i = 0; i < detours.getPossibleDetours().size();i++){
    if (representation.getBitString()[i]){
      visitedDetours.add(detours.getPossibleDetours().get(i));
    }
  }
  return visitedDetours.stream().sorted(Comparator.comparing(Detours.Detour::getPosition)).toList();
}

\end{lstlisting}


\subsubsection{Code Smell: Code Comments}

Der Kommentar 'remove runaways' in der Methode percentileandaverage der Klasse SpeedHeuristics ist ein Anzeichen dafür, dass die Funktionalität des Codes ohne den Kommentar nicht verständlich ist.


\begin{lstlisting}

for (double d:paceValues) {
  if (i + 1 > paceValues.size() * 0.25 && i < paceValues.size() * 0.9) // remove runaways
  sum += d;
  instances ++;
  i++;
}
\end{lstlisting}

Der gewählte Lösungsweg ist die Auslagerung der Funktionalität in eine Methode mit sprechendem Namen auszulagern.

\begin{lstlisting}
...
var consideredPaceValues = removeRunawaysfromSortedList(paceValues, 0.25, 0.9);
...

private static List<Double> removeRunawaysfromSortedList(List<Double> input, double lowerBound, double upperBound){
  if(lowerBound < 0 || lowerBound > 1 || upperBound < 0 || upperBound > 1){
     throw new RuntimeException("Bounds in removeRunaways should be a value between 0 and 1");
  }
  int length = input.size();
  return input.subList((int)(length*lowerBound),(int)Math.ceil(length*upperBound));
}

\end{lstlisting}

\subsection{2 Refactorings}

\subsubsection{1. Refactoring: Rename Class}

\href{https://github.com/reichert-p/GPXrechner/commit/7947597903210e73647ffbc20d07e6b5b257cff2\#diff-a3b6bacc77a02d2a1746cbbbd776a467bf3fcd7920bfefbef1e892af10172e50 }{a3b6bac}

Die Klasse EvolutionaryDist implementiert die evolutionäre Suche nach lokalen Optima nach dem Hillclimbing-Verfahren.
Das Refactoring ist die Umbenennung der Klasse EvolutionaryDist zu Hillclimbing, da der Name besser zum Inhalt der Klasse passt.

\begin{figure}[H]
  \centering
  \includesvg[inkscapelatex = false, width = 90pt]{Chapters/Chapter7/EvolutionaryDist.svg}
  \caption{UML Diagramm der Klasse EvolutionaryDist}
\end{figure}

\begin{figure}[H]
  \centering
  \includesvg[inkscapelatex = false, width = 90pt]{Chapters/Chapter7/HillClimbing.svg}
  \caption{UML Diagramm der Klasse unter dem Namen HillClimbing}
\end{figure}

\subsubsection{2. Refactoring: Extract Method}

Die Methode generateGPX zur Erstellung einer .gpx Datei aus gespeicherten Touren und Tracks besteht aus 35 Zeilen langem Spaghetticode. 
In den Commits 
\href{https://github.com/reichert-p/GPXrechner/commit/59f9045a2ac73496111bba87c35016c2b26108e2}{59f9045}
und 
\href{https://github.com/reichert-p/GPXrechner/commit/7d57943bb5fcdf2d23e76f05c0157f7753f6c05e}{7d57943} 
wurde mithilfe von Method Struktur in den Code gebracht, die Länge der einzelnen Methoden reduziert und das Verständnis des Codes durch Sprechenden Methodennamen verbessert.

\begin{figure}[H]
  \centering
  \includesvg[inkscapelatex = false, width = 90pt]{Chapters/Chapter7/XMLGenSpaghetti.svg}
  \caption{UML Diagramm der Klasse XMLGeneratorImplementation vor dem Refactoring}
\end{figure}

\begin{figure}[H]
  \centering
  \includesvg[inkscapelatex = false, width = 90pt]{Chapters/Chapter7/XMLGenRefactored.svg}
  \caption{UML Diagramm der Klasse XMLGeneratorImplementation vor dem Refactoring}
\end{figure}