\section{Kapitel 2: Clean Architecture}

\subsection{Was ist Clean Architecture?}

Clean Architecture ist ein Softwarearchitekturmuster, das darauf abzielt, Code klar zu organisieren und leicht wartbar, testbar und erweiterbar zu machen. Hierfür werden die Bestandteile einer Anwendung in verschiedenen hierarchischen Schichten gekapselt, sodass äußere Schichten von inneren abhängen können, jedoch nicht umgekehrt. Tiefere Schichten sind langlebiger als äußere Schichten.


\subsection{Analyse der Dependency Rule}

\subsubsection{Positiv-Beispiel: Dependency Rule}

Das Positivbeispiel ist die Klasse \textit{GetDistance}, die eine Implementation eines \textit{Instruction} Interfaces ist. Sie liegt in der Anwendungsschicht.
Sie implementiert den Anwendungsfall, dass Benutzer die Strecke eines Weges erfahren wollen.

Um den Weg zu erhalten, für den die Strecke berechnet werden soll, greift sie auf den \textit{State} auf der Anwendungsschicht zu.
Die eigentliche Berechnung findet mithilfe der Klasse \textit{DistanceCalculator} statt, welche in der Domain-Schicht liegt.
 Die Strecke wird über das \textit{UserOutput} Interface mitgeteilt, welches in der Anwendungsschicht liegt.
 Damit hat die Klasse \textit{GetDistance} lediglich Abhängigkeiten auf die eigene und tiefere Schichten und erfüllt die Dependency Rule.

\begin{figure}[H]
  \centering
  \includesvg[inkscapelatex = false, width = 300pt]{Chapters/Chapter2/positiv1}
  \caption{Abhängigkeiten der Klasse GetDistance}
\end{figure}


\subsubsection{Negativ-Beispiel: Dependency Rule}

Die Dependency Rule wird beim Zugriff auf die Klasse \textit{GPXToTour} verletzt, die in der Plugin-Schicht liegt und aus GPX Dateien ein Tour Objekt generiert. Sie wird von der Klasse \textit{ReadPath} verwendet, welche in der Anwendungsschicht liegt. Somit besteht eine Abhängigkeit von der inneren Anwendungsschicht zur äußeren Plugin-Schicht, was eine Verletzung der Dependency Rule darstellt.

\begin{figure}[H]
  \centering
  \includesvg[inkscapelatex = false, width = 300pt]{Chapters/Chapter2/negativ}
  \caption{Abhängigkeiten auf die Klasse GPXToTour}
\end{figure}

\subsubsection{Schicht: Domain Code}

Die Klasse \textit{DistanceCalculator} ist dafür zuständig, verschiedene Distanzen zwischen Orten oder einer chronologischen Abfolgen von Orten im Sexagesimalsystem zu berechnen.
Die (angemessen genaue) Berechnung von Distanzen im Sexagesimalsystem basieren auf grundlegenden geometrischen Zusammenhängen, die sich in absehbarer Zeit nicht ändern werden. 
Diese Berechnungen sind grundlegend für alle Auswertungen von Daten, die im Sexagesimalsystem gespeichert sind, so wie beispielsweise GPS-Daten im GPS Exchange Format(GPX).

\begin{figure}[H]
  \centering
  \includesvg[inkscapelatex = false, width = 100pt]{Chapters/Chapter2/ClassUML1}
  \caption{UML Diagramm der Klasse DistanceCalculator}
\end{figure}

\subsubsection{Schicht: Plugins}

Die Klasse \textit{ConsoleParsing} ist dafür zuständig, verschiedene Formen Input von Benutzern zu erfassen.
Dies umfasst beispielsweise den Pfad zu GPX Dateien, die Wahl einer Sportart oder Geschwindigkeit sowie die Eingabe einer Zeit.
Somit stellt die Klasse einen wesentlichen Bestandteil der Benutzerschnittstelle dar.
Es wäre denkbar, die Klasse durch eine grafische Benutzerschnittstelle zu ersetzen.

\begin{figure}[H]
  \centering
  \includesvg[inkscapelatex = false, width = 100pt]{Chapters/Chapter2/ClassUML2}
  \caption{UML Diagramm der Klasse ConsoleParsing}
\end{figure}


