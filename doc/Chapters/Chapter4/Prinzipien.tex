\section{Kapitel 4: Weitere Prinzipien}

\subsection{Analyse GRASP: Geringe Kopplung}


\subsubsection{Positiv-Beispiel}

Die Abbildung zeigt das Positiv-Beispiel für geringe Kopplung.
Die verschiedenen Klassen im Package \textit{Instructions} implementieren die Abläufe verschiedener Anwendungsfälle. Dabei müssen Entscheidungen von Anwendern während der Nutzung berücksichtigt werden.
Die Klasse \textit{ConsoleParsing} nimmt Nutzereingaben von der Konsole entgegen.
 Das Interface \textit{UserInput} entkoppelt die \textit{Instructions} von den Benutzereingaben über die Konsole.
Damit wird eine direkte Abhängigkeit der Anwendungslogik vermieden.
 Das erleichtert eine Änderung auf eine nicht-Konsolen-Applikation, was einen Vorteil der darstellt. 

\begin{figure}[H]
  \label{fig:GKGut}
  \centering
  \includesvg[inkscapelatex = false, scale = 0.25]{Chapters/Chapter4/GeringeKopplungGut}
  \caption{UML Diagramm des Interfaces \textit{UserInput}}
\end{figure}

\subsubsection{Negativ-Beispiel}

Die Abbildung zeigt das Negativ-Beispiel für geringe Kopplung.
Die Klasse \textit{Hillclimbing} implementiert eine Annäherung\footnote{Insbesondere bei kleinen Datenmengen ist auch ein Erreichen möglich. Dies kann aber aufgrund der Komplexität des Problems (oder möglicherweise meinem fehlenden Verständnis) nicht garantiert werden} an eine optimale Lösung zur Mitnahme von wichtigen Wegpunkten, die eine Abweichung vom eigentlichen Wege erfordern.
\\

Die Klasse \textit{Detours} stellt alle in Betracht zu ziehenden Umwege dar. 
Die Klasse \textit{Representation} stellt in Kombination mit Detours eine Lösung des Problems in Form eines Bitstrings dar.
Eine \textit{EvaluationFunction} bewertet Lösungen des Problems.
\textit{MovementSpeed} stellt die Geschwindigkeit dar, mit der sich auf dem Weg bewegt wird.

 Die Klasse \textit{Hillclimbing} besitzt zwar eine geringe Kopplung zu \textit{EvaluationFunction}, \textit{MovementSpeed} und den zugrundeliegenden Wegpunkten über das \textit{Location} Interface.
 Allerdings besteht eine starke Kopplung zwischen der \textit{Hillclimbing} Klasse und sowohl der Repräsentation der Lösungen und den mögliche Umwegen (Realisiert in den Klassen \textit{Representation} und \textit{Detours}), da die Klasse \textit{Hillclimbing} von der Detailumsetzung der Umwege und insbesondere der Darstellung von Lösungen abhängt.

\begin{figure}[H]
 \label{fig:GKBad}
  \centering
  \includesvg[inkscapelatex = false, scale = 0.25]{Chapters/Chapter4/GeringeKopplungBad}
  \caption{UML Diagramm der Klasse \textit{Hillclimbing}}
\end{figure}

Die Kopplungen könnten gelöst werden, indem zwischen \textit{Detours} und \textit{Hillclimbing} sowie zwischen \textit{Representation} und \textit{Hillclimbing} jeweils ein Interface genutzt würde.
Diese Interfaces müssten dann auch in die \textit{EvaluationFunction} übergeben werden, da auch diese eine hohe Kopplung zu den beiden Klassen hat.
Mithilfe dieser Umsetzung könnte sowohl eine neue Repräsentation für Lösungen sowie eine neue Umsetzung möglicher Lösungsbestandteilen aufgrund der geringeren Kopplung implementiert werden.

Dies würde allerdings zu einigen Problemen führen. Beide Klassen nur miteinander sinnvoll von einer Bewertungsfunktion auswertbar sind, da beide Bestandteil einer Lösung sind und damit eine hohe Kohäsion haben. 
Sinnvoller wäre wohl, das Prinzip der hohen Kohäsion anzuwenden, und die so erstellte Lösungskombination gering zu koppeln.


\subsection{Analyse GRASP: Hohe Kohäsion}

Die Klasse \textit{Trackpoint} referenziert eine Länge, eine Breite und eine Höhe über dem Meeresspiegel.
Als Gesamtheit repräsentiert ein \textit{Trackpoint} also einen präzisen Ort auf der Erde.
Die Kohäsion ist also sehr hoch, da genau diese Werte zusammen ein wohldefinierten Punkt im dreidimensionalen Raum abbilden.


\begin{figure}[H]
 \label{fig:HighKohesion}
  \centering
  \includesvg[inkscapelatex = false, scale = 0.5]{Chapters/Chapter4/Highkohesion}
  \caption{UML Diagramm der Klasse \textit{Hillclimbing}}
\end{figure}


\subsection{Don’t Repeat Yourself (DRY)}

Im Commit \href{https://github.com/reichert-p/GPXrechner/commit/8ffd648d794563fea2c8662debe12ca1277b1b3e#diff-3be7b5bf425b2c9794272cdc879a4e9b0ed23bab55acbdf71756fbda1356ff80}{8ffd648} wird der duplizierte Code aus den Klassen \textit{ElevationProfile} und \textit{SpeedProfile} aufgelöst.

In beiden Klassen werden Diagramme berechnet. Es gibt Berechnungen, die unabhängig von dem Inhalt des Profils durchgeführt werden müssen. Ein Beispiel hierfür ist die Methode \textit{normalize}, welche den Wertebereich der Profile auf Zahlen zwischen 0 und 1 normiert.


\begin{lstlisting}[caption={Sich wiederholender Code vor dem Commit}]

// SpeedProfile.java (Zeile 39)

speeds = normalize(speeds,min,max);
		
// (Zeile 50-58)

private List<Double> normalize(List<Double> list,double min, double max){
	double diff = max - min;
	for (int i = 0; i < list.size(); i++) {
		double val = list.get(i);
		double normalizedVal = (val - min) / diff;
		list.set(i,normalizedVal);
	}
	return list;
}   

// ElevationProfile.java (Zeile 37)

heights = normalize(heights,min,max);
		
// (Zeile 48-56)

private List<Double> normalize(List<Double> list,double min, double max){
	double diff = max - min;
	for (int i = 0; i < list.size(); i++) {
		double val = list.get(i);
		double normalizedVal = (val - min) / diff;
		list.set(i,normalizedVal);
	}
	return list;
}

\end{lstlisting}


\begin{lstlisting}[caption={Angewandtes DRY-Prinzip nach dem Commit}]

// SpeedProfile.java (Zeile 39)

speeds = ProfileCalculation.normalize(speeds,min,max);
				
// ElevationProfile.java (Zeile 37)

heights = ProfileCalculation.normalize(heights.stream().map(e->e.getValue()).toList(), min.getValue(), max.getValue());

// ProfileCalculation.java (Zeile 6-14)

public static List<Double> normalize(List<Double> list, double min, double max){
	double diff = max - min;
	for (int i = 0; i < list.size(); i++) {
		double val = list.get(i);
		double normalizedVal = (val - min) / diff;
		list.set(i,normalizedVal);
	}
	return list;
}

\end{lstlisting}