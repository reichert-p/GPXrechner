\section{Kapitel 5: Unit Tests}

\subsection{10 Unit Tests}

\begin{tabularx}{16cm}{l X}

Unit Test & Beschreibung \\
ElevationGainTest\#addElevation & Test auf korrekte Summierung von a Auf- und Abstieg \\
ElevationGainTest\#getManhattenNorm & Test auf korrekte Berechnung der Manhattennorm eines Elevationgains \\
DistanceCalculatorTest\#calc3dDistance & Test auf korrekt genuge heuristische Berechnung im dreidimensionalen Raum \\
DistanceCalculatorTest\#calcElevationGain & Test auf korrekte Berechnung einer Höhendifferenz zwische 2 Punken und eines gesamten GPX Tracks \\
ProfileCalculationTest\#split & Test auf die Korrekte Aufteilung von Wegpunkten in Balken für Profile  \\
ProfileCalculationTest\#normalize & Test auf die Korrekte Normalisierung von Datenpunkten für die Erstellung von Profilen \\
SpeedHeuristicsTest\#calculateTime & Test auf die Korrekte Auswertung der tatsächlich benötigten Zeit aus Toursegmenten \\
ElevationProfileTest\#getProfile & Test auf die Korrekte erstellung einer Matrix die ein Höhenprofil repräsentiert \\
SpeedCalculatorTest\#predictPMSSingle &  Test auf die Korrekte erstellung eines Personal Movement Speeds (PMS) aus einer gegangenen Tour \\
SpeedHeuristicsTest\#getClimbingHeuristic & Test auf genau genuge heuristische Berechnung einer Geschwindigkeit mit der Steigungen bezwungen werden \\

\end{tabularx}

\subsection{ATRIP: Automatic}

Automatic wurde realisiert indem per rechtem Mausklick auf Verzeichnis src/test und Auswahl der Option 'Run 'Tests in test'' alle Tests ausgeführt werden.

\subsection{ATRIP: Thorough}

\subsubsection{Positivbeispiel}
ProfileCalculationTest der die Grundfunktionen für die Erstellung von Profilen bereitstellt. Das Thoroughe dabei ist dass alle Fehlerfälle abgedeckt sind die Mathematisch auftreten können, wie etwa die Eingabe von Nullparametern oder die Unterrepräsentation von Daten.

\begin{lstlisting}

@Test
    void normalize(){
        List<Double> list = new ArrayList<>();
        list.add(-10.0);
        list.add(0.0);
        list.add(10.0);
        list = ProfileCalculation.normalize(list);
        assertEquals(0,list.get(0));
        assertEquals(0.5,list.get(1));
        assertEquals(1,list.get(2));
    }

    @Test
    void normalizeFlatDiff(){
        List<Double> list = new ArrayList<>();
        list.add(4.0);
        list.add(4.0);
        list.add(4.0);
        list = ProfileCalculation.normalize(list);
        assertEquals(1,list.get(0));
        assertEquals(1,list.get(1));
        assertEquals(1,list.get(2));
    }

    @Test
    void split(){
        int[] sections = ProfileCalculation.split(22,5);
        assertEquals(5,sections[0]);
        assertEquals(5,sections[1]);
        assertEquals(4,sections[2]);
        assertEquals(4,sections[3]);
        assertEquals(4,sections[4]);
    }

    @Test
    void splitTooSmallGranularity(){
        int[] sections = ProfileCalculation.split(22,25);
        assertEquals(1,sections[19]);
        assertEquals(1,sections[20]);
        assertEquals(1,sections[21]);
        assertEquals(0,sections[22]);
    }

\end{lstlisting}

\subsubsection{Negativbeispiel}
Instructions für die Konsole

Dadurch, dass diese Klassen die Äußerste Schicht im Sinne der Clean Architecture darstellen und aufgrund der Anforderung auf Fokus ausserhalb der User Experience liegt sind diese Klassen weder besonders komplex noch in einem vollständig ausgearbeiteten Zustand, welcher außerhalb einer Konsolenanwendung läge.

Da die Hauptlogik in der Interaktion mit dem Benutzer liegt, wurden hier manuelle acceptence Tests angewandt %richtiger begriff?

\subsection{ATRIP: Professional}

\subsubsection{Positivbeispiel}

The distanceCalculatortest Tests the correct interpretation of Distances from a real Track with an reduced amount of Waypoints.
 Those waypoints are created in an extra Helper Class, since It's creation is not in the responsibility of the DistanceCalculator to be tested. They are Called before each Test, which ensures the tests are isolated from one another.
Additionally, there are no big code smells in the Classes to be tested. 

\begin{lstlisting}
   Track mountainTrack;

    @BeforeEach
    void init() {
        mountainTrack = GetTracks.getMountainTrack();
    }

    @Test
    void calc2dDistance() {
        Distance distanceToFirstHut2D = DistanceCalculator.calc2dDistance(mountainTrack.getOrderedLocations().get(0), mountainTrack.getOrderedLocations().get(1));
        assertEquals(308, distanceToFirstHut2D.getValue(), 1);

        Distance distanceOfWholeTrack2D = DistanceCalculator.calc2dDistance(mountainTrack);
        assertEquals(7795, distanceOfWholeTrack2D.getValue(), 10);
    }

    @Test
    void calc3dDistance() {
        Distance distanceToFirstHut2D = DistanceCalculator.calc3dDistance(mountainTrack.getOrderedLocations().get(0), mountainTrack.getOrderedLocations().get(1));
        assertEquals(310, distanceToFirstHut2D.getValue(), 1);

        Distance distanceOfWholeTrack3D = DistanceCalculator.calc3dDistance(mountainTrack);
        assertEquals(8315, distanceOfWholeTrack3D.getValue(), 10);
    }

    @Test
    void calcElevationGain() {
        ArrayList<Location> locations = mountainTrack.getOrderedLocations();
        ElevationGain uphillSection = DistanceCalculator.calcElevationGain(locations.get(1), locations.get(2)); //uphill
        assertEquals(569, uphillSection.getUp(), 1);
        assertEquals(0, uphillSection.getDown(), 1);

        ElevationGain downhillSection = DistanceCalculator.calcElevationGain(locations.get(4), locations.get(5)); //downhill
        assertEquals(0, downhillSection.getUp(), 1);
        assertEquals(539, downhillSection.getDown(), 1);

        ElevationGain wholeTrack = DistanceCalculator.calcElevationGain(mountainTrack); //whole Track
        assertEquals(1346, wholeTrack.getUp(), 1);
        assertEquals(1493, wholeTrack.getDown(), 1);
    }

    @Test
    void calcAltitudeData() throws InsufficientDataException {
        List<Location> locations = mountainTrack.getOrderedLocations();
        Elevation lowestPoint = DistanceCalculator.calcMinAlt(locations);
        assertEquals(1096, lowestPoint.getValue(), 1);

        Elevation highestPoint = DistanceCalculator.calcMaxAlt(locations);
        assertEquals(2589, highestPoint.getValue(), 1);

        Elevation averageAltitude = DistanceCalculator.calcAvgAlt(locations);
        assertEquals(1778, averageAltitude.getValue(), 1);
    }
\end{lstlisting}

\subsubsection{Negativbeispiel}

Ein negativbeispiel ist die Methode evaluationFunction der Klasse EvolutionaryDistTest. Aufgrund der hohen komplexität des Algorithmus und der Eingabeparameter wird auf höchstem Level quasi eine Instruction mit gegebenen Usereingaben ausgeführt und der Status der Klasse danach überprüft. Allerdings werden auch DOMParser und Bewertungsfunktion mitgetested, was nicht professionell ist.

\begin{lstlisting}
   Track mountainTrack;

    @BeforeEach
    void init() {
        mountainTrack = GetTracks.getMountainTrack();
    }

    @Test
    void calc2dDistance() {
        Distance distanceToFirstHut2D = DistanceCalculator.calc2dDistance(mountainTrack.getOrderedLocations().get(0), mountainTrack.getOrderedLocations().get(1));
        assertEquals(308, distanceToFirstHut2D.getValue(), 1);

        Distance distanceOfWholeTrack2D = DistanceCalculator.calc2dDistance(mountainTrack);
        assertEquals(7795, distanceOfWholeTrack2D.getValue(), 10);
    }

    @Test
    void calc3dDistance() {
        Distance distanceToFirstHut2D = DistanceCalculator.calc3dDistance(mountainTrack.getOrderedLocations().get(0), mountainTrack.getOrderedLocations().get(1));
        assertEquals(310, distanceToFirstHut2D.getValue(), 1);

        Distance distanceOfWholeTrack3D = DistanceCalculator.calc3dDistance(mountainTrack);
        assertEquals(8315, distanceOfWholeTrack3D.getValue(), 10);
    }

    @Test
    void calcElevationGain() {
        ArrayList<Location> locations = mountainTrack.getOrderedLocations();
        ElevationGain uphillSection = DistanceCalculator.calcElevationGain(locations.get(1), locations.get(2)); //uphill
        assertEquals(569, uphillSection.getUp(), 1);
        assertEquals(0, uphillSection.getDown(), 1);

        ElevationGain downhillSection = DistanceCalculator.calcElevationGain(locations.get(4), locations.get(5)); //downhill
        assertEquals(0, downhillSection.getUp(), 1);
        assertEquals(539, downhillSection.getDown(), 1);

        ElevationGain wholeTrack = DistanceCalculator.calcElevationGain(mountainTrack); //whole Track
        assertEquals(1346, wholeTrack.getUp(), 1);
        assertEquals(1493, wholeTrack.getDown(), 1);
    }

    @Test
    void calcAltitudeData() throws InsufficientDataException {
        List<Location> locations = mountainTrack.getOrderedLocations();
        Elevation lowestPoint = DistanceCalculator.calcMinAlt(locations);
        assertEquals(1096, lowestPoint.getValue(), 1);

        Elevation highestPoint = DistanceCalculator.calcMaxAlt(locations);
        assertEquals(2589, highestPoint.getValue(), 1);

        Elevation averageAltitude = DistanceCalculator.calcAvgAlt(locations);
        assertEquals(1778, averageAltitude.getValue(), 1);
    }
    
\end{lstlisting}

\subsection{Code Coverage}
Aktuell liegt die Testabdeckung bei 78\% class coverage und 73\% line coverage. Der Grund hierfür ist hauptsächlich die geringe Testabdeckung der äußeren Schichten im Sinne der clean architecture die in den Verzeichnissen Application und Interfaces liegen, während die Verzeichnisse mit den komplizierten Berechnungen und der Grundstruktur bei 91 bzw. 94\% line coverage liegen. Die restlichen Prozente lassen sich durch auslassen trivialster Testfälle, etwa getter und setter, erklären.

\subsection{Fakes und Mocks}
 % TODO fake/mock Objekte nutzen
 
 \subsubsection{1. Mockobjekt: DirectWayHeuristik}
 
 Hier wurde beim Test der Erstellung von Umwegen ein Mockobjekt für die Bestimmung der Zeit für die Umleitung verwendet. Dies ist Sinnvoll, da hier die Korrekte Initialisierung der Umwege an der richtigen Stelle getestet werden soll und nicht die Bestimmung der Zeit, die man benötigt um sich eine Gewisse Strecke zu bewegen.
 
 \begin{figure}[H]
  \centering
  \includesvg[inkscapelatex = false, width = 200pt]{Chapters/Chapter5/DetoursTestMock.svg}
  \caption{UML Diagramm des Mocks für DetoursTests}
\end{figure}

 
 \subsubsection{2. Mockobjekt: EvaluationFunction}
 
 Hier wurde beim Test des Hillclimbingalgorithmus die Bewertung der einzelnen Tests gemockt. Die Bewertungsfunktion bewertet alle Lösungen gleich gut bis auf die Lösung wahr,wahr,falsch, die bevorzugt bewertet wird. Dies macht deutlich, dass diese Lösung als Lokales (und globales) Optimum beim Hillclimbingalgorithmus herauskommen soll, da es das einzige existierende Optimum ist. Das Mock Objekt ist sinnvoll, da die Bewertungsfunktion in anderen Tests abgedeckt ist und es lediglich um das Korrekte finden Lokaler Optimums geht bei der Hillclimbing-Klasse.
 
  \begin{figure}[H]
  \centering
  \includesvg[inkscapelatex = false, width = 200pt]{Chapters/Chapter5/Hillclimbing.svg}
  \caption{UML Diagramm des Mocks für eine Bewertungsfunktion}
\end{figure}
