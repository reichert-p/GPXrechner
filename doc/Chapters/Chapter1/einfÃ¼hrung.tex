\section{Kapitel 1:Einführung}


\subsection{Übersicht über die Applikation}

Der GPX Rechner ist ein Programm zur Auswertung von GPX Dateien. Mit seiner Hilfe können Strecke, Dauer und Höhenprofil geplanter Touren vorhergesagt werden und bereits gegangene Touren ausgewertet werden auf Konsistenz der Geschwindigkeit und Geschwindigkeitsheuristiken für weitere Planungen.

Ein besonderes Feature bei geplanten Strecken ist die Aufteilung dieser anhand wichtiger Punkte die regelmäßig besucht werden müssen, wie etwa Unterkünfte oder Wasserquellen.

Die Benutzung der Applikation erfolgt indem man seine Dateien im Format dateiname.gpx in den Projektpfad kopiert. Mit starten der Main() Methode wird das Terminal gestartet und man wird aufgefordert einen Befehl einzugeben. Bei falscher Eingabe (bzw. "hilfe") werden alle verfügbaren Befehle aufgelistet.



\subsection{Wie startet man die Applikation?}

Zum Starten der Applikation führt man die Main-Methode unter \textit{src/GPXrechner/Main.java} aus und geht in das aufkommende Eingabefeld. Hier wird man aufgefordert, eine Instruktion einzugeben. Gibt man eine nicht zulässigen Befehl bzw. 'help' ein, so bekommt man eine Übersicht über alle möglichen Instruktionen. Gewöhnlich fängt man mit der instruction 'load gpx' an, um eine Tour/Track als Basis für seine weiteren Instruktionen zu bekommen.


\subsection{Wie testet man die Applikation?}

Zum Testen der Applikation führt man alle Tests im Directory \textit{src/test/} aus, in IntelliJ etwa mit rechtem Mausklick auf das Directory und Auswahl von 'Run 'Tests in 'test''.